\documentclass[12pt]{paper}
\textwidth=17cm
\textheight=23.5cm
\usepackage[utf8]{inputenc}
\usepackage[spanish]{babel}
\usepackage{amsmath,amssymb,exscale,graphicx}
%\input epsf
\parskip 0.3cm
\usepackage{bm}


\title{\begin{center}Bayesian Statistics and Machine Learning Workshop 2023\end{center}}
\subtitle{\begin{center}\Large Introuction to Classification Methods\\ Martín Onetto \end{center}}

\begin{document}
\maketitle


\topmargin -2.0cm
\oddsidemargin -0.2cm
\evensidemargin -0.2cm

\vspace{-80pt}
%This practice is intended for your own exercise and {\bf does not} need to be turned in. 

\section{Questions}

\begin{enumerate}
\item Cómo es el proceso de clasificación?
\item Decribir el proceso que se definen las fronteras de decisión para la regresión lineal, logistica, LDA y QDA
\item ¿Cómo se podría \textit{bayesianizar} los métodos vistos?
\item ¿Qué cambiaría en los resultados obtenidos?
\item ¿Cómo afectaría el número de datos de cada clase a las fronteras de las etiquetas?
\end{enumerate}

\subsection{Problems}

\textbf{Gaussians}

Simular datos de 3 Gaussianas multivariabdas con $\Sigma = diag(1,1)$ y $\mu_{1} = (0,0)$,$\mu_{2} =(1,1)$ y $\mu_{3} = (-1,-1)$. Cada guassiana ahora representa una clase $Y = k$ con $k \in \{1,2,3\}$
\begin{enumerate}
\item Armar las fronteras de decisión con cada uno de los métodos vistos.
\item Para cada método evalular el Error rate de la clasiicación. 
\item Encontrar la dirección de máxima varianza según el método de Fisher y hacer un plot de las gaussianas sobre esa dirección.
\end{enumerate}
\newpage

\textbf{Vowel recognition}

Cargar la base de datos vowel.csv que contiene una variable target $y = k \in \{1,\dots,11\}$. Cada $y_{k}$ corresponde a una vocal estas son: 
\begin{verbatim}
| vowel |  word  | vowel |  word   | 
+-------+--------+-------+---------+
|  i    |  heed  |  O    |  hod    |
|  I    |  hid   |  C:   |  hoard  |
|  E    |  head  |  U    |  hood   |
|  A    |  had   |  u:   |  who'd  |
|  a:   |  hard  |  3:   |  heard  |
|  Y    |  hud   |       |         |
+-------+--------+-------+---------+
\end{verbatim}
Para determinar estas vocales tenemos datos $X$ con dimensión $p = 10$ que corresponden a procesamientos auditivos de diferentes frecuencias de mediciones de voz. Para este conjunto de datos:
\begin{enumerate}
\item Armar las fronteras de decisión con regresión logistica, LDA y QDA.
\item Para cada método evalular el Error rate de la clasiicación. 
\item Encontrar las dos direcciones de máxima varianza según el método de Fisher y hacer un scatter plot de los datos con sus respectivas clases en esas direcciones.
\item Para el caso de la regresión logistica, calcular la posterior de los parámetros y su varianza. Con al información de la posterior queremos ver cómo se ve afectada la linea de decisión respeto a usar el máximun likelihoo. Para esto elegir dos clases $g_{i}$ y $g_{j}$ y calcular su frontera marginalizando sobre la posterior de $\beta$ es decir:
\begin{equation}
\log \frac{P(G=i|X)}{P(G=j|X)} = \frac{\int P(G=i|X,\beta) P(\beta|X)d\beta}{\int P(G=j|X,\beta) P(\beta|X)d\beta}
\end{equation}
tomar como posterior de $\beta$ su aproxmiación gaussiana por el método de Laplace.
\item Trazar las curvas fronteras entre las dos clases elegidas que resultan de samplear 1000 $\beta^{s}$ de la posterior inferida. 
\item Discutir la diferencia entre considerar la incerteza en los parámetros y no hacerlo.
\item Repetir el procedimiento anterior para QDA pero sólo hacer inferencia bayesiana sobre los $\mu$s, es decir seguir estimando cada matriz de covarianza como\newline $ \Sigma_{k} = \frac{1}{N_{k}-p-1} \sum_{i=1}^{N_{k}} (y^{k}_{i}-\bar{y}^{k}) (y^{k}_{i}-\bar{y}^{k})^{T}$. 
\end{enumerate}

\pagestyle{empty}



\end{document}
