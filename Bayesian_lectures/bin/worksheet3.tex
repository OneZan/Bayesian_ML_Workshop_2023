\documentclass[12pt]{paper}
\textwidth=17cm
\textheight=23.5cm
\usepackage[utf8]{inputenc}
\usepackage[spanish]{babel}
\usepackage{amsmath,amssymb,exscale,graphicx}
%\input epsf
\parskip 0.3cm
\usepackage{bm}


\title{\begin{center}Bayesian Statistics and Machine Learning Workshop 2023\end{center}}
\subtitle{\begin{center}\Large Estimación de parámetros y la distribución gaussiana\\ Martín Onetto \end{center}}

\begin{document}
\maketitle


\topmargin -2.0cm
\oddsidemargin -0.2cm
\evensidemargin -0.2cm

\vspace{-80pt}
%This practice is intended for your own exercise and {\bf does not} need to be turned in. 

\section{Questions}

\begin{enumerate}
\item Discutir el significado de \textit{sufficient statistic}.
\item ¿Cómo es la dependencia de la incerteza en los parámetros a medida que aumentamos el número de datos?
\item ¿Qué relación tiene la incerteza de los parámetros con la calidad del ajuste del modelo? 
\item Discutir la influencia de la prior en las inferencia de los parámetros.
\end{enumerate}
\section{Problems}
En el caso donde no conocemos el valor de $\sigma$ en un modolo gaussiano la inferencia la hacemos sobre los parámetros $\mu,\sigma$. En el caso de $\mu$ ya vimos que la prior menos informativa que podemos poner es una constante. En el caso de $\sigma^{2}$ por ser un parámetro de escala se considera que debe ser uniforme en $\log(\sigma)$ que resulta en $P(\sigma) \prop \frac{1}{\sigma}$. Es decir que:
\begin{equation}
P(\mu,\sigma^{2}) \propto \frac{1}{\sigma^{2})
\end{equation}

\begin{enumerate}
\item 
\item Incorporar ahora los datos de 20 tiradas $c$ y  15 $s$. Qué sucede con la forma de la disribución? Como cambia su ancho? Puede vincular la posción del máximo con alguna información de los datos?

\end{enumerate}

\pagestyle{empty}

\end{document}
