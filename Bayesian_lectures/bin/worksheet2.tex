\documentclass[12pt]{paper}
\textwidth=17cm
\textheight=23.5cm
\usepackage[utf8]{inputenc}
\usepackage[spanish]{babel}
\usepackage{amsmath,amssymb,exscale,graphicx}
%\input epsf
\parskip 0.3cm
\usepackage{bm}


\title{\begin{center}Bayesian Statistics and Machine Learning Workshop 2023\end{center}}
\subtitle{\begin{center}\Large Prior, Likelihood and Posterior Distribution\\ Martín Onetto \end{center}}

\begin{document}
\maketitle


\topmargin -2.0cm
\oddsidemargin -0.2cm
\evensidemargin -0.2cm

\vspace{-80pt}
%This practice is intended for your own exercise and {\bf does not} need to be turned in. 

\section{Questions}

\begin{enumerate}
\item Discutir el significado de \textit{prior},\textit{likelihood} y \textit{posterior} distribution.
\item En el esquema de actualización de información, que hipótesis es la que hac que se llegue al mismo resultado si incorporamos los datos individualmente o todos a la vez.
\item ¿Cómo se manifiesta el aumento de información sobre nuestra confianza de la moneda? 
\item ¿Por qué se hace cero la probabilidad de que $\theta$ valga 1 cuando incorporamos el dato donde salió seca?
\end{enumerate}
\section{Problems}

\begin{enumerate}
\item Graficar la prior y ver cómo evoluciona la posterior a medida de que se incorporan los datos $ D = \{c,c,s\}$.

\item Incorporar ahora los datos de 20 tiradas $c$ y  15 $s$. Qué sucede con la forma de la disribución? Como cambia su ancho? Puede vincular la posción del máximo con alguna información de los datos?

\end{enumerate}

\pagestyle{empty}

\end{document}
